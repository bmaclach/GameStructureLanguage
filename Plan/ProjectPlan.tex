\documentclass[12pt]{article}

\usepackage{fullpage}
\usepackage{hyperref}
\usepackage{booktabs}

\hypersetup{
	bookmarks=true,        % show bookmarks bar?
	colorlinks=true,       % false: boxed links; true: colored links
	urlcolor=blue          % color of external links
}

\title{COMPSCI 6TB3 Project Plan}

\author{Brooks MacLachlan}

\begin{document}
	\maketitle
	
	\section{Project Description}
	
	The goal of this project is to develop a grammar and DSL for describing 
	game structures. For the purposes of this project, a "game" is simply an 
	activity involving one or more players with some concept of "winning". 
	
	Variabilities between games include the number of rounds, number of 
	players, player affiliations (are there teams, or is everyone playing 
	individually?), the win condition, and game "events" that dictate how the 
	game progresses. Some examples of potential events are updates to a 
	player's affiliation or score, and events could be triggered by results of 
	player performance-based challenges or player decisions. The language 
	should be expressive enough to  describe all of the following examples of 
	game structures:
	
	\begin{itemize}
		\item Tennis: a game consisting of 7 rounds between 2 players where the 
		winner is the player who wins the majority of rounds.
		\item Baseball: a game consisting of 9 rounds between 2 teams where the 
		winning team is the one with the greatest cumulative score across all 9 
		rounds.
		\item Survivor (based on the television show with the same name): a 
		game with many players consisting of a series of rounds where a player 
		is eliminated in each round by majority vote and the winner is the last 
		player standing (many different variations of this formula would be 
		possible and expressible by the grammar)
		\item Many more
	\end{itemize}
	
	Note that the language is describing only the game structure, not the 
	details of how the game works. For example, the mechanisms behind 
	performance-based challenges are not of concern here. What matters is how 
	the results of those challenge impact how the game progresses. The details 
	of \textit{how} a player wins a round of tennis do not matter, but the 
	result of winning a round of tennis is that the player moves closer to 
	winning the entire game, and that is the kind of structure that will be 
	defined by the language.
	
	\section{Resources}
	
	The compiler for the DSL will be written in Haskell. Tests will be written 
	with the assistance of \href{https://hspec.github.io/}{HSpec, a testing 
	framework for Haskell}. \href{https://www.haskell.org/haddock/}{Haddock} 
	will be used to generate documentation for the software. The DSL will be 
	compiled into Python 3 code. Various television programs will be used as 
	inspiration for game structures to use as examples to assist in development 
	of the language. These include Survivor, Big Brother, The Genius, and 
	sports.
	
	\section{Division of Work}
	
	Brooks MacLachlan is solely responsible for completing the project.
	
	\section{Weekly Schedule}
	
	\begin{tabular}{l p{13cm}}
		\toprule
		Week & Tasks to complete\\
		\midrule
		March 18-24 & Fully define the grammar and write examples\\
		March 25-31 & Write Python programs for some 
		of the examples, to get an idea of what the final, compiled product 
		should look like. Develop the compiler in Haskell, writing tests and 
		documentation along the way.\\
		April 1-7 & Design and complete poster\\
		April 8-15 & Final touches and print poster\\
		April 16-17 & Poster presentation\\
		\bottomrule
	\end{tabular}
	
\end{document}