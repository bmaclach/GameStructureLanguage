\documentclass[a1paper, 25pt, colspace=5mm, 
subcolspace=0mm, 
blockverticalspace=5mm, innermargin=5mm]{tikzposter}

\usepackage[rounded]{syntax}
\usepackage{listings}
\usepackage{url}

\title{A language and compiler for game structures}
\author{Brooks MacLachlan}
\institute{McMaster University}
\date{\today}

\usetheme{Desert}

\useblockstyle{Basic}

\usebackgroundstyle{Empty}

\begin{document}
	\maketitle
	\begin{columns}[fragile]
		\column{0.5}
		\block{Background}{\begin{itemize}
				\item A ``game'' is an activity involving some number of 
				players and a way to ``win''
				\begin{itemize}
					\item Ex. sports, reality show games
				\end{itemize}
				\item Infinitely many games are possible, but running most 
				games would require a human ``host''
				\item An easy-to-understand DSL for describing game structures
				and generating programs to run the game could reduce the 
				need for human hosts for game administration and eliminate 
				possibility of host "fixing" the game for a certain player
			\end{itemize}}
		\block{Related Work}{\begin{itemize}
				\item Existing game DSLs are focused on video games or computer 
				players ([1], for example)
				\item Brantsteele is a website for simulating games [2]
				\item Few playable game variations available on websites such 
				as Tengaged [3] and Zwooper [4]
				\item Online Reality Games (ORG) designed and run by human 
				hosts are often played on social media platforms (see [5])
			\end{itemize}}
		\block{Development Information}{\textbf{The Program}
			\begin{itemize}
				\item 1500+ lines of Haskell code
				\item Target language is Python
			\end{itemize}
			\textbf{Documentation}
			\begin{itemize}
				\item All Haskell functions formally documented with Haddock
			\end{itemize}
			\textbf{Tests}
			\begin{itemize}
				\item Parser, PreCompiler, and Compiler are fully unit-tested 
				using the HSpec framework in Haskell
				\item Over 320 test cases in total!
				\item 6 additional full example games act as 
				integration tests
				\begin{itemize}
					\item Tennis, baseball, Survivor, Big Brother, The Genius, 
					original
				\end{itemize}
			\end{itemize}}
		\begin{subcolumns}
			\subcolumn{0.45}
			\block{References}{\small
			$[1]$ Love, N., Hinrichs, T., Haley, D., Schkufza, E., Genesereth, 
			M. (2008). General Game Playing: Game Description Language 
			Specification. \textit{Stanford Logic Group}.\\
			$[2]$ Brantsteele. \url{https://brantsteele.com}\\
			$[3]$ Tengaged. \url{https://tengaged.com}\\
			$[4]$ Zwooper. \url{https://zwooper.com}\\
			$[5]$ OnlineSurvivor. 
			\url{https://www.reddit.com/r/OnlineSurvivor/}}
			\subcolumn{0.55}
			\block{Modules}{\includegraphics{ModuleHierarchy}}
		\end{subcolumns}
		\column{0.5}
		\block{Subset of the Grammar}{
			\begin{grammar}
				<game> ::= `Players:' <teamList> `Rounds:' <roundList> `Win:' 
				<winCondition>
				
				<competition> ::= [`scored'] [`team'] `competition between' 
				<identifierList>
				
				<decision> ::= `vote by' <identifierList> `between' 
				<identifierList> [`including self'] | ...
				
				<affiliationUpdate> ::= (`add' | `remove') 
				<name> | ...
				
				<counterUpdate> ::= `set' <name> `to' <value> | ...
				
		\end{grammar}}
	\block{Example - DSL to Python}{\textbf{Snippet of game description}:
		\lstinputlisting{exGameDesc.txt}
		\textbf{Parser}: Uses the Parsec library
		\lstinputlisting{exParser.hs}
		\textbf{AST node}:
		\lstinputlisting{exAST.hs}
		\textbf{Compiler}:
		\lstinputlisting{exCompiler.hs}
		\textbf{Final Python code}:
		\lstinputlisting{exPython.py}
	}
	\block{Conclusion and Future Work}{\begin{itemize}
			\item The domain of game structures can be captured by a DSL, 
			including well-known and completely original game structures
			\item The generated code in its current form is not particularly 
			useful, improve with user interface, online-support, additional 
			features such as game advantages, more conditional 
			possibilities within rounds, variable names for phases, rounds, or 
			tiebreakers
		\end{itemize}
	}
	\end{columns}
\end{document}